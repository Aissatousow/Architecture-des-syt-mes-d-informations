\documentclass[12pt,a4paper,titlepage]{article}

\usepackage[utf8]{inputenc}
\usepackage[T1]{fontenc}
\usepackage[francais]{babel}
\usepackage{datetime}  


\usepackage[letterpaper,top=2cm,bottom=2cm,left=3cm,right=3cm,marginparwidth=1.75cm]{geometry}

% Useful packages
\usepackage{amsmath}
\usepackage{graphicx}
\usepackage[colorlinks=true, allcolors=blue]{hyperref}

\title{Architecture des Systèmes d'Information {}}
\author{Aïssatou SOW\\
   Matricule : 000565711\\
   Université Libre de Bruxelles\\
   Master en Sciences et Technologie de l'Information et de la Communication\\
   Bruxelles - Belgique\\
   \texttt{aïssatou.sow@ulb.be}}
\date{\today}

\maketitle

\begin{document}
\maketitle

\clearpage
\vspace*{\fill}
\begin{center}
\begin{minipage}{.6\textwidth}
\begin{center}
\huge {\textbf {Les réseaux sociaux numériques comme outils de travails pour les
étudiants  }} 
\end{center}
\end{minipage}
\end{center}
\vfill % equivalent to \vspace{\fill}
\clearpage

\tableofcontents


\newpage

\section{Introduction}
A l’heure où un nombre considérable de personnes et d’entreprises utilisent les réseaux
sociaux comme moyen de communication, cette ressource est extrêmement peu utilisée dans
le cadre des institutions à caractère social. \newline
La transformation digitale a une portée considérable tant pour les entreprises et le monde du
travail que pour la société dans son ensemble. Cette révolution numérique est si importante
que des spécialistes la comparent à la naissance de l’imprimerie il y a plus de cinq siècles.
Dans cette nouvelle ère, les canaux numériques se multiplient et leurs usages montent en
puissance. Les réseaux sociaux sont devenus des outils incontournables de communication.
Les organisations doivent aujourd’hui tirer avantages des opportunités digitales pour
développer leur notoriété, leur chiffre d’affaires, adapter leur culture d’entreprise et fidéliser
leurs collaborateurs. Augmentation de la visibilité de la marque, partage des actualités fidélisation des clients et des employés, recrutement … Facebook, Twitter, LinkedIn...
présentent de nombreux atouts pour l’entreprise. \newline
L’internet est actuellement le plus grand réseau informatique sur notre planète. On peut
l’appeler réseaux sociaux. L’internet ne se limite plus aux universités, aux industries et aux
gouvernements. Aujourd’hui tout le monde l’utilise, car chaque particulier peut maintenant se
joindre à ces réseaux sociaux. L’internet permet d’échanger les informations en toute liberté.
En même temps, on observe le développement dynamique des réseaux sociaux qui deviennent
plus populaires et plus utilisés. \newline
En conséquence, les entreprises introduisent les nouveaux outils de la promotion en profitant
des avantages donnés par les nouvelles technologies. \newline
L’objectif de mon travail de recherche consiste à faire une analyse sur l’apprentissage des
étudiants à travers les réseaux sociaux numérique. \newline
Aujourd’hui plus qu’avant, nous vivons dans un monde virtuel, rendu possible par le
développement sans précédent des technologies de l’information et de la communication (TIC). \newline
Depuis quelques années maintenant, nous assistons à une métaphore de l’internet. Ce
changement à un nom « le web ». Les réseaux sociaux en ligne s’inscrivent dans ce contexte. \newline
Les réseaux sociaux s’adaptent à toutes les thématiques possibles ; recherche d’emploi ou de
développement de business, rencontre entre individus, échange d’information autour d’un
centre d’intérêt commun, partage de contenus multimédia ou de la musique, etc. Mais les
réseaux sociaux proposent nombreux enjeux comme la gestion de son identité en ligne ainsi
que sa réputation. Les réseaux sociaux offrent une visibilité importante. De ce fait, il est
facilement possible de contrôler les informations circulant sur le Net. Il est également possible
d’avoir de nombreuses opportunités par rapport aux ambitions des utilisateurs. \newline
\begin{itemize}
    \item Quels usages et opportunités ?
    \item Est-ce qu’il y a un impact sur l’utilisation ?
    \item Quelle est l’utilité d’un réseau social ?
    \item Les réseaux sociaux numérique est-elle un outil d’apprentissage pour les étudiants ?
\end{itemize}

\section{Hypothèses :}
Un réseau social pourrait être un facilitateur d’apprentissage pour les étudiants. 

\subsection{Définition des réseaux sociaux :}
\begin{figure}[htp]
    \centering
    \includegraphics[width=15cm]{Image1.jpg}
    \caption{Réseaux sociaux}
    \label{fig:galaxy}
\end{figure}
Dans le domaine des technologies, un réseau social consiste en un service permettant de
regrouper diverses personnes afin de créer un échange sur un sujet particulier ou non. En
quelque sorte, le réseau social trouve ses origines dans les forums, groupes de discussion et
salons de chat introduits dès les premières heures d'Internet. Depuis le début des années 2000,
la présence des réseaux sociaux, également appelés réseaux communautaires, devient de plus
en plus importante et tend à se multiplier selon diverses caractéristiques. Les premiers réseaux
sociaux de grande envergure (Myspace et Facebook) se sont positionnés en tant que services
généralistes sur lesquels chacun peut partager le contenu de son choix, quel qu'en soit le sujet,
avec ses contacts. \newline
Une définition plus moderne d’un réseau social est apparue en 2004 comme « un ensemble de
relations entre un ensemble d’acteurs ». Cet ensemble peut être organisé (c’est le cas d’une
entreprise) ou non (comme un réseau d’amis) et ces relations peuvent être de nature fort diverse
(pouvoir, échanges de cadeaux, conseil, etc.), spécialisées ou non, symétriques ou non. Il s’agit d’un élément immatériel qui définit l’interaction entre des éléments ou des personnes qui font
partir d’un même ensemble en vue de leurs points communs, matériels ou immatériels. \newline
Les réseaux sociaux existaient bien avant l’internet. Un réseau social n’est en effet rien d’autre
qu’un groupe de personnes ou d’organisations reliées entre elles par les échanges sociaux
qu’elles entretiennent.6 Aujourd’hui le réseau que constitue internet a démultiplié ces réseaux
sociaux et interaction et les a dotés d’une toute nouvelle puissance. Pour survivre, un réseau
social doit engendrer une interdépendance entre ses membres. Ceux-ci ont besoin de partager
leurs expériences et d’obtenir le feedback des autres membres, autrement dit leurs réactions.
Ces expériences peuvent être sous forme d’informations, d’articles, de vidéos ou encore
d’images. \newline


\subsection{L’utilité des réseaux sociaux, usage et opportunités :}
Chaque réseau social a ses spécificités et les avantages des réseaux sociaux sont
multiples. Néanmoins, tous partagent des bénéfices communs, en voici quelques-uns. \newline
Les réseaux sociaux sont des excellents moyens de développer votre visibilité digitale et donc
votre notoriété. C’est devenu le canal de diffusion d’informations et d’actualités des entreprises.
Ils permettent ainsi d’accroitre votre portée et de « casser » les barrières pour atteindre plusieurs
cibles à la fois en utilisant le principe de viralité : partenaires, fournisseurs, candidats,
collaborateurs… et même les clients. D’ailleurs, plusieurs entreprises comme la SNCF ou les
opérateurs téléphoniques utilisent les réseaux sociaux (Twitter ou Facebook) comme
plateformes de service client. Cela leur permet d’avoir un contact direct avec eux. Mais
attention, cela suppose d’être extrêmement organisé en termes d’éléments de communication,
de gestion des commentaires, des questions et de réputation. Pour ajouter une source de trafic
additionnel, vous pouvez inclure le lien de votre site à vos publications. \newline
Mais, il y a des inconvénients des réseaux sociaux. Ils peuvent être très abusifs. Il y a des gens
qui ont besoin d’être sur ces sites toutes les heures ou toutes les minutes ! Cela peut créer de
graves problèmes pour eux. Beaucoup d’étudiants trouvent qu’il est difficile à étudier car ils
sont trop distraits par leur Facebook. Souvent, les résultats des examens sont pauvres à cause
de cela. D’autres personnes peuvent prendre du retard au travail parce qu’ils sont trop occupés
avec les réseaux sociaux. Un problème inquiétant avec les réseaux sociaux est le
cyberintimidation. C’est à ce moment que les utilisateurs disent des choses méchantes et
blessantes sur d’autres utilisateurs. Il peut avoir de graves conséquences. \newline

\subsection{Réseaux sociaux numériques comme outils d’apprentissage :}
D’une révolution causée par l’évolution des Tics. L’avènement des outils du Web2.0
tels que les médias sociaux, permettent aux étudiants de créer du contenu, échanger
des idées et partager des connaissances en accès libre. La participation aux médias
sociaux crée un environnement d’apprentissage collaboratif et communicatif pour les
étudiants en leur offrant des opportunités pour des discussions et des interactions avec leurs
pairs. \newline
Une étude réalisée par des chercheurs en technologies de l’éducation à l’Université du
Minnesota montre en effet qu’il peut s’agir de formidables outils éducatifs. Des lycéens
américains entre 16 et 18 ans ont été observés et interrogés pendant six mois sur leur rapport
à Internet. Près de la totalité d’entre eux déclarent utiliser Internet. Plus de 80 % dit se
connecter de chez soi et les trois quarts ont un profil sur un réseau social comme Myspace,
Facebook ou d’autres forums de discussion. \newline
Et à la question que vous apportez la fréquentation de ces sites ? La principale réponse est
l’acquisition de compétences technologiques. Les autres motivations citées étant notamment
la créativité, le fait d’être ouvert à la nouveauté et à des opinions différentes. \newline

\end{document}