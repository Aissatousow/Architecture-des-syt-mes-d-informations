\documentclass[12pt,a4paper,titlepage]{article}

\usepackage[utf8]{inputenc}
\usepackage[T1]{fontenc}
\usepackage[francais]{babel}
\usepackage{datetime}  


\usepackage[letterpaper,top=2cm,bottom=2cm,left=3cm,right=3cm,marginparwidth=1.75cm]{geometry}

% Useful packages
\usepackage{amsmath}
\usepackage{graphicx}
\usepackage[colorlinks=true, allcolors=blue]{hyperref}

\title{Architecture des Systèmes d'Information {}}
\author{Aïssatou SOW\\
   Matricule : 000565711\\
   Université Libre de Bruxelles\\
   Master en Sciences et Technologie de l'Information et de la Communication\\
   Bruxelles - Belgique\\
   \texttt{aïssatou.sow@ulb.be}}
\date{\today}

\maketitle

\begin{document}
\maketitle

\clearpage
\vspace*{\fill}
\begin{center}
\begin{minipage}{.6\textwidth}
\begin{center}
\huge {\textbf {Les réseaux sociaux numériques comme outils de travails pour les
étudiants  }} 
\end{center}
\end{minipage}
\end{center}
\vfill % equivalent to \vspace{\fill}
\clearpage

\tableofcontents


\newpage

\section{Introduction}
A l’heure où un nombre considérable de personnes et d’entreprises utilisent les réseaux
sociaux comme moyen de communication, cette ressource est extrêmement peu utilisée dans
le cadre des institutions à caractère social. \newline
La transformation digitale a une portée considérable tant pour les entreprises et le monde du
travail que pour la société dans son ensemble. Cette révolution numérique est si importante
que des spécialistes la comparent à la naissance de l’imprimerie il y a plus de cinq siècles.
Dans cette nouvelle ère, les canaux numériques se multiplient et leurs usages montent en
puissance. Les réseaux sociaux sont devenus des outils incontournables de communication.
Les organisations doivent aujourd’hui tirer avantages des opportunités digitales pour
développer leur notoriété, leur chiffre d’affaires, adapter leur culture d’entreprise et fidéliser
leurs collaborateurs. Augmentation de la visibilité de la marque, partage des actualités fidélisation des clients et des employés, recrutement … Facebook, Twitter, LinkedIn...
présentent de nombreux atouts pour l’entreprise. \newline
L’internet est actuellement le plus grand réseau informatique sur notre planète. On peut
l’appeler réseaux sociaux. L’internet ne se limite plus aux universités, aux industries et aux
gouvernements. Aujourd’hui tout le monde l’utilise, car chaque particulier peut maintenant se
joindre à ces réseaux sociaux. L’internet permet d’échanger les informations en toute liberté.
En même temps, on observe le développement dynamique des réseaux sociaux qui deviennent
plus populaires et plus utilisés. \newline
En conséquence, les entreprises introduisent les nouveaux outils de la promotion en profitant
des avantages donnés par les nouvelles technologies. \newline
L’objectif de mon travail de recherche consiste à faire une analyse sur l’apprentissage des
étudiants à travers les réseaux sociaux numérique. \newline
Aujourd’hui plus qu’avant, nous vivons dans un monde virtuel, rendu possible par le
développement sans précédent des technologies de l’information et de la communication (TIC). \newline
Depuis quelques années maintenant, nous assistons à une métaphore de l’internet. Ce
changement à un nom « le web ». Les réseaux sociaux en ligne s’inscrivent dans ce contexte. \newline
Les réseaux sociaux s’adaptent à toutes les thématiques possibles ; recherche d’emploi ou de
développement de business, rencontre entre individus, échange d’information autour d’un
centre d’intérêt commun, partage de contenus multimédia ou de la musique, etc. Mais les
réseaux sociaux proposent nombreux enjeux comme la gestion de son identité en ligne ainsi
que sa réputation. Les réseaux sociaux offrent une visibilité importante. De ce fait, il est
facilement possible de contrôler les informations circulant sur le Net. Il est également possible
d’avoir de nombreuses opportunités par rapport aux ambitions des utilisateurs. \newline
\begin{itemize}
    \item Quels usages et opportunités ?
    \item Est-ce qu’il y a un impact sur l’utilisation ?
    \item Quelle est l’utilité d’un réseau social ?
    \item Les réseaux sociaux numérique est-elle un outil d’apprentissage pour les étudiants ?
\end{itemize}

\end{document}