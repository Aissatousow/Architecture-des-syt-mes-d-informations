\documentclass[12pt,a4paper,titlepage]{article}

\usepackage[utf8]{inputenc}
\usepackage[T1]{fontenc}
\usepackage[francais]{babel}
\usepackage{datetime}  


\usepackage[letterpaper,top=2cm,bottom=2cm,left=3cm,right=3cm,marginparwidth=1.75cm]{geometry}

% Useful packages
\usepackage{amsmath}
\usepackage{graphicx}
\usepackage{quotchap}
\usepackage[colorlinks=true, allcolors=blue]{hyperref}

\title{Architecture des Systèmes d'Information {}}
\author{Aïssatou SOW\\
   Matricule : 000565711\\
   Université Libre de Bruxelles\\
   Master en Sciences et Technologie de l'Information et de la Communication\\
   Bruxelles - Belgique\\
   \texttt{aïssatou.sow@ulb.be}}
\date{\today}

\maketitle

\begin{document}
\maketitle

\clearpage
\vspace*{\fill}
\begin{center}
\begin{minipage}{.6\textwidth}
\begin{center}
\huge {\textbf {Les réseaux sociaux numériques comme outils de travails pour les
étudiants  }} 
\end{center}
\end{minipage}
\end{center}
\vfill % equivalent to \vspace{\fill}
\clearpage

\tableofcontents


\newpage

\section{Introduction}
A l’heure où un nombre considérable de personnes et d’entreprises utilisent les réseaux
sociaux comme moyen de communication, cette ressource est extrêmement peu utilisée dans
le cadre des institutions à caractère social. \newline
La transformation digitale a une portée considérable tant pour les entreprises et le monde du
travail que pour la société dans son ensemble. Cette révolution numérique est si importante
que des spécialistes la comparent à la naissance de l’imprimerie il y a plus de cinq siècles.
Dans cette nouvelle ère, les canaux numériques se multiplient et leurs usages montent en
puissance. Les réseaux sociaux sont devenus des outils incontournables de communication.
Les organisations doivent aujourd’hui tirer avantages des opportunités digitales pour
développer leur notoriété, leur chiffre d’affaires, adapter leur culture d’entreprise et fidéliser
leurs collaborateurs. Augmentation de la visibilité de la marque, partage des actualités fidélisation des clients et des employés, recrutement … Facebook, Twitter, LinkedIn...
présentent de nombreux atouts pour l’entreprise. \newline
L’internet est actuellement le plus grand réseau informatique sur notre planète. On peut
l’appeler réseaux sociaux. L’internet ne se limite plus aux universités, aux industries et aux
gouvernements. Aujourd’hui tout le monde l’utilise, car chaque particulier peut maintenant se
joindre à ces réseaux sociaux. L’internet permet d’échanger les informations en toute liberté.
En même temps, on observe le développement dynamique des réseaux sociaux qui deviennent
plus populaires et plus utilisés. \newline
En conséquence, les entreprises introduisent les nouveaux outils de la promotion en profitant
des avantages donnés par les nouvelles technologies. \newline
L’objectif de mon travail de recherche consiste à faire une analyse sur l’apprentissage des
étudiants à travers les réseaux sociaux numérique. \newline
Aujourd’hui plus qu’avant, nous vivons dans un monde virtuel, rendu possible par le
développement sans précédent des technologies de l’information et de la communication (TIC). \newline
Depuis quelques années maintenant, nous assistons à une métaphore de l’internet. Ce
changement à un nom « le web ». Les réseaux sociaux en ligne s’inscrivent dans ce contexte. \newline
Les réseaux sociaux s’adaptent à toutes les thématiques possibles ; recherche d’emploi ou de
développement de business, rencontre entre individus, échange d’information autour d’un
centre d’intérêt commun, partage de contenus multimédia ou de la musique, etc. Mais les
réseaux sociaux proposent nombreux enjeux comme la gestion de son identité en ligne ainsi
que sa réputation. Les réseaux sociaux offrent une visibilité importante. De ce fait, il est
facilement possible de contrôler les informations circulant sur le Net. Il est également possible
d’avoir de nombreuses opportunités par rapport aux ambitions des utilisateurs. \newline
\begin{itemize}
    \item Quels usages et opportunités ?
    \item Est-ce qu’il y a un impact sur l’utilisation ?
    \item Quelle est l’utilité d’un réseau social ?
    \item Les réseaux sociaux numérique est-elle un outil d’apprentissage pour les étudiants ?
\end{itemize}

\section{Hypothèses :}
Un réseau social pourrait être un facilitateur d’apprentissage pour les étudiants. 

\subsection{Définition des réseaux sociaux :}
\begin{figure}[htp]
    \centering
    \includegraphics[width=15cm]{Image1.jpg}
    \caption{Réseaux sociaux}
    \label{fig:galaxy}
\end{figure}
Dans le domaine des technologies, un réseau social consiste en un service permettant de
regrouper diverses personnes afin de créer un échange sur un sujet particulier ou non. En
quelque sorte, le réseau social trouve ses origines dans les forums, groupes de discussion et
salons de chat introduits dès les premières heures d'Internet. Depuis le début des années 2000,
la présence des réseaux sociaux, également appelés réseaux communautaires, devient de plus
en plus importante et tend à se multiplier selon diverses caractéristiques. Les premiers réseaux
sociaux de grande envergure (Myspace et Facebook) se sont positionnés en tant que services
généralistes sur lesquels chacun peut partager le contenu de son choix, quel qu'en soit le sujet,
avec ses contacts. \newline
Une définition plus moderne d’un réseau social est apparue en 2004 comme « un ensemble de
relations entre un ensemble d’acteurs ». Cet ensemble peut être organisé (c’est le cas d’une
entreprise) ou non (comme un réseau d’amis) et ces relations peuvent être de nature fort diverse
(pouvoir, échanges de cadeaux, conseil, etc.), spécialisées ou non, symétriques ou non. Il s’agit d’un élément immatériel qui définit l’interaction entre des éléments ou des personnes qui font
partir d’un même ensemble en vue de leurs points communs, matériels ou immatériels. \newline
Les réseaux sociaux existaient bien avant l’internet. Un réseau social n’est en effet rien d’autre
qu’un groupe de personnes ou d’organisations reliées entre elles par les échanges sociaux
qu’elles entretiennent.6 Aujourd’hui le réseau que constitue internet a démultiplié ces réseaux
sociaux et interaction et les a dotés d’une toute nouvelle puissance. Pour survivre, un réseau
social doit engendrer une interdépendance entre ses membres. Ceux-ci ont besoin de partager
leurs expériences et d’obtenir le feedback des autres membres, autrement dit leurs réactions.
Ces expériences peuvent être sous forme d’informations, d’articles, de vidéos ou encore
d’images. \newline


\subsection{L’utilité des réseaux sociaux, usage et opportunités :}
Chaque réseau social a ses spécificités et les avantages des réseaux sociaux sont
multiples. Néanmoins, tous partagent des bénéfices communs, en voici quelques-uns. \newline
Les réseaux sociaux sont des excellents moyens de développer votre visibilité digitale et donc
votre notoriété. C’est devenu le canal de diffusion d’informations et d’actualités des entreprises.
Ils permettent ainsi d’accroitre votre portée et de « casser » les barrières pour atteindre plusieurs
cibles à la fois en utilisant le principe de viralité : partenaires, fournisseurs, candidats,
collaborateurs… et même les clients. D’ailleurs, plusieurs entreprises comme la SNCF ou les
opérateurs téléphoniques utilisent les réseaux sociaux (Twitter ou Facebook) comme
plateformes de service client. Cela leur permet d’avoir un contact direct avec eux. Mais
attention, cela suppose d’être extrêmement organisé en termes d’éléments de communication,
de gestion des commentaires, des questions et de réputation. Pour ajouter une source de trafic
additionnel, vous pouvez inclure le lien de votre site à vos publications. \newline
Mais, il y a des inconvénients des réseaux sociaux. Ils peuvent être très abusifs. Il y a des gens
qui ont besoin d’être sur ces sites toutes les heures ou toutes les minutes ! Cela peut créer de
graves problèmes pour eux. Beaucoup d’étudiants trouvent qu’il est difficile à étudier car ils
sont trop distraits par leur Facebook. Souvent, les résultats des examens sont pauvres à cause
de cela. D’autres personnes peuvent prendre du retard au travail parce qu’ils sont trop occupés
avec les réseaux sociaux. Un problème inquiétant avec les réseaux sociaux est le
cyberintimidation. C’est à ce moment que les utilisateurs disent des choses méchantes et
blessantes sur d’autres utilisateurs. Il peut avoir de graves conséquences. \newline

\subsection{Réseaux sociaux numériques comme outils d’apprentissage :}
D’une révolution causée par l’évolution des Tics. L’avènement des outils du Web2.0
tels que les médias sociaux, permettent aux étudiants de créer du contenu, échanger
des idées et partager des connaissances en accès libre. La participation aux médias
sociaux crée un environnement d’apprentissage collaboratif et communicatif pour les
étudiants en leur offrant des opportunités pour des discussions et des interactions avec leurs
pairs. \newline
Une étude réalisée par des chercheurs en technologies de l’éducation à l’Université du
Minnesota montre en effet qu’il peut s’agir de formidables outils éducatifs. Des lycéens
américains entre 16 et 18 ans ont été observés et interrogés pendant six mois sur leur rapport
à Internet. Près de la totalité d’entre eux déclarent utiliser Internet. Plus de 80 % dit se
connecter de chez soi et les trois quarts ont un profil sur un réseau social comme Myspace,
Facebook ou d’autres forums de discussion. \newline
Et à la question que vous apportez la fréquentation de ces sites ? La principale réponse est
l’acquisition de compétences technologiques. Les autres motivations citées étant notamment
la créativité, le fait d’être ouvert à la nouveauté et à des opinions différentes. \newline
\subsection{Mais pour quel usage ?}
Très rapidement, les services éducatifs ont développé des ateliers qui utilisent
l’informatique. Plusieurs pistes étaient et restent possibles. Du simple apprentissage du
traitement de texte pour la réalisation de curriculum vitae, on peut aussi aller vers la
conception d’affiches ou de plaquettes, de journaux, de périodiques, pour la création
d’événements. Ces derniers temps, le passage du Brevet de sécurité routière se fait désormais
à partir d’un logiciel, face auquel les jeunes sont seuls à travailler, et qui énonce les résultats
dès la fin de l’ultime question du test. Il est également possible de se lancer dans le dessin sur
ordinateur. Et même, certains se sont mus en véritables techniciens de maintenance, en tentant
de donner une seconde jeunesse à du matériel de récupération. Ce qui parfois permettait aussi
d’allier à la mise en place technique une visée humanitaire, les machines ainsi reconstruites
partant vers des contrées plus techniquement défavorisées.
\subsubsection{Les forums des étudiants sur internet}
On ne peut pas dire que le forum soit une « nouvelle » technologie, car il existe depuis près de
30 ans... De fait, le premier réseau de forums de discussions électroniques, connu sous le nom
de Usenet (Unix User Network) a été créé en 1979 par des étudiants de Caroline du Nord aux
États-Unis. Ce réseau est toujours utilisé aujourd'hui. Sans entrer dans les détails, on peut dire
que le réseau de serveurs Usenet permet l'échange de messages entre les membres d'un groupe
pouvant être éparpillés géographiquement. De nombreux groupes de discussion (newsgroups)
sur des sujets très variés utilisent toujours cette technologie pour communiquer. Toutefois, la
création d'un tel groupe est plutôt compliquée et nécessite un grand nombre de participants. De
plus, la confidentialité est problématique puisque tout le monde peut avoir accès aux messages
publiés dans ce réseau. \newline
\subsection{Définition :}
Un forum de discussion électronique est un lieu d'échange asynchrone entre des
membres d'un groupe plus ou moins nombreux. Les participants y publient des messages et
répondent à ceux des autres, ce qui forme des fils de discussion. En général, le contenu d'un
forum est durable, c'est-à-dire que les messages publiés y sont conservés pendant une longue
période et les participants y ont accès en tout temps. Dans un forum, les messages sont
généralement présentés sous une forme chronologique (les messages apparaissent selon leur
ordre de publication dans le temps) ou hiérarchique (chaque message est rattaché au message
antérieur auquel il répond). Souvent, un forum compte différentes sections réservées à des
discussions sur des sujets, des thèmes, des questions ou des tâches spécifiques à réaliser. Par exemple,
dans un cours de littérature, un forum pourrait être divisé en quelques sections selon les différentes
œuvres à étudier. Dans chaque section, les messages porteraient sur l'œuvre spécifiée. \newline

\subsubsection{En Exemple}
Avant d'aller plus loin, prenons quelques instants pour présenter un exemple concret
d'un petit forum Web, créé en Phibbs, dans lequel a eu lieu un débat permettant d'explorer
différents types de garde pour les enfants d'âge préscolaire. \newline
Sur la page d'accueil du forum, après s'être identifié, le participant se retrouvait avec une liste
des différentes sections du forum ainsi que leur description. \newline
\begin{figure}[htp]
    \centering
    \includegraphics[width=15cm]{Image3.png}
    \label{fig:galaxy}
\end{figure}
En sélectionnant une des sections, le participant pouvait aller y voir les sujets des fils de
discussion. Voici par exemple, les sujets de la section « Le débat », où le débat avait lieu.\newline
\begin{figure}[htp]
    \centering
    \includegraphics[width=15cm]{Image2.png}
    \label{fig:galaxy}
\end{figure}

\subsection{Le contexte pédagogique :}
Le forum de discussion électronique peut être utilisé dans un cadre éducatif pour faire
un encadrement en ligne des étudiants, c'est-à-dire pour fournir un soutien individuel d'ordre
affectif (motivation, cheminement à suivre, etc.) et d'ordre cognitif (lié au contenu du cours) à
chacun des apprenants afin qu'ils réalisent les apprentissages souhaités. Il permet aussi de créer
un environnement facilitant le travail d'équipe et d'intervenir, comme enseignant ou tuteur, pour
favoriser la collaboration. Son utilisation permet aux étudiants d'avoir rapidement des réponses
à leurs questions et de contacter leur enseignant, leur tuteur ou leurs pairs. Elle peut entraîner
la création de liens d'appartenance à un groupe, ce qui réduit, pour les étudiants à distance
surtout, le sentiment d'isolement et contribue à la persévérance.\newline
Le forum électronique est un outil asynchrone particulièrement propice aux activités qui
impliquent des discussions entre plusieurs personnes. Contrairement au clavardage, à la
visioconférence et aux autres outils synchrones, il permet aux étudiants de réfléchir et
d’effectuer des recherches avant d'intervenir dans une discussion ou d'apporter leur contribution
à une tâche à réaliser. Il permet aux plus timides et à ceux qui vivent plus d'insécurité de prendre
le temps de s'exprimer et de le faire au moment où ils le souhaitent. Cet outil a aussi l'avantage
de conserver les messages sur de longues périodes. Cela permet aux étudiants de retourner lire
des messages antérieurs afin de mieux comprendre certains éléments de contenu et à
l'enseignant de vérifier si les étudiants ont tous contribué à un travail d'équipe réalisé avec cet
outil. De plus, contrairement à des outils comme le courrier électronique et les listes de diffusion
qui permettent aussi d'échanger des messages textes, le forum conserve et rend accessibles tous
les messages des fils de discussion, et cela, en tout temps et de partout si on a accès à Internet.\newline
\subsection{Est-ce qu’il y a un impact sur l’utilisation ?}
Quand on utilise des outils technologiques sociaux tels que le forum électronique, on
n'a pas, comme enseignant, le même rôle que lorsqu'on fait un exposé magistral devant sa
classe. Comme l'explique Charlier (2002), l'enseignant qui anime un forum a plusieurs fonctions : il doit être facilitateur, leader, «ange gardien», conseiller pour le processus de
collaboration, aide pour la prise de décision, aide technique. Les étudiants attendent de lui qu'il
soit visiblement impliqué dans le travail, que ses rétroactions soient rapides, qu'il soit
«démocratique» (par opposition à «autocratique» ou «laisser-faire»), qu'il s'adapte à chaque
étudiant individuellement. D’AELE (2002) précise qu'il a quatre rôles :
 
\begin{itemize}
    \item \textbf{Rôle social : } : créer un environnement accueillant et amical qui facilite
l'apprentissage, encourager les étudiants, les amener à travailler ensemble dans un but
commun.

    \item \textbf{Rôle d'organisation :} : « intervention managériale » touchant entre autres
l'agenda, l'organisation du travail, etc.).
    \item \textbf{Rôle pédagogique :} : faciliter l'apprentissage entre autres en attirant l'attention sur
les éléments importants, en suscitant des questionnements, etc...
    \item \textbf{Rôle technique :} : aider les étudiants à utiliser le matériel technologique. 
\end{itemize}

Dans le cas d'un forum de discussion pédagogique, l'animation est nécessaire pour jouer tous
ces rôles et pour atteindre les buts qu'on se fixe. De ce fait, un facteur de succès du forum
pédagogique est son animation.\newline
L'essence même d'un forum est la discussion ! Vous pouvez donc y avoir différents types de
discussion et d'échange avec vos étudiants. Par exemple, M. Steve Boucher, enseignant en
géographie au Cégep régional de Lanaudière, expliquait au dernier colloque de l'APOP, qu'il
utilise le forum dans le cadre d'un cours pour poser chaque semaine des questions d'actualité à
ses étudiants qui les préparent à la prochaine période de cours. Il utilise également cet outil pour
amener ses étudiants à échanger à distance avec un expert.\newline
\section{Autres réseaux sociaux numériques et leurs spécificités utilisés par les étudiants : }

La présence
d’une association sur les réseaux sociaux ne relève presque pas d’un choix, mais plutôt d’une nécessité.
En revanche, pour que cette décision soit productive et diffuse le meilleur de vous-même, il s’agit d’être
pertinent quant à la qualité de votre présence digitale. En effet, tous les médias sociaux ne se valent pas.
Chacun fait même valoir toutes ses spécificités. Le choix du réseau sur lequel être présent est
directement lié à l’ADN de votre association et aux types de contenus que vous voulez pousser et à la
présence ou non de vos publics. Réfléchissez bien aux médias que vous considérez pertinents pour
véhiculer vos contenus et toucher vos publics. \newline
\begin{itemize}
    \item \textbf{Facebook} :Plus de 26 millions d’utilisateurs en France. C’est un réseau social qui permet de partager
tout type de contenu (textes, vidéos, images, être en direct, partager des liens, etc.) et animer une
conversation avec vos publics.
    \item \textbf{Twitter} : Près de 6,8 millions de comptes actifs en France. Le public qui l’utilise est généralement
jeune de 15 à 34 ans. Près de 70-80% de journalistes sont sur le twitter et la plupart des hommes
politiques et autorités publiques, les acteurs, sportifs, etc. ont un compte twitter. Il est devenu une des
principales sources d’information en temps réel. Tous les événements sont visibles et commentés à la
seconde, ce qui confère à cette plateforme autant de puissance que de risques. C’est une plateforme de
micro-blogging, ça veut dire que vos posts sont limités en caractère ; vous avez 160 symboles pour faire
un message.
    \item \textbf{Linkedin} : Réseau professionnel par excellence, Linkedin recense 6 millions de comptes actifs en
France. En plus de travailler votre marque employeur, de part, notamment, la visibilité qu’apportent
voscollaborateurs, Linkedin permet de diffuser du contenu de qualité relatif à des sujets directement ou
moins directement reliés à votre activité, auprès de professionnels et de prospects de vos secteurs
d’activité.
    \item \textbf{YouTube} :Depuis sa création en 2005 et son rachat dans la foulée par Google, la plateforme n’en finit
pas d’imposer sa domination sur les contenus vidéos. Aujourd’hui, YouTube compte 1 milliard
d’utilisateurs dans le monde et 22 millions en France. Une prééminence qui en fait un réseau 2
incontournable dans la vie quotidienne des internautes. Deux chiffres suffisent à prouver son poids dans
l’écosystème digital. Le nombre d’heures de visionnage mensuelles sur YouTube augmente de 50 %
chaque année tandis que 300 heures de vidéo sont mises en ligne chaque minute sur le réseau. Une
plateforme utile pour faire partager et promouvoir vos vidéos.
 \item \textbf{Pinterest et Instagram} : Applications concurrentes permettant de diffuser de l’information sous
forme de visuels qui connaissent une forte progression en termes d’usage par les internautes en France.
Très utile donc pour une association ayant une activité visuelle et pour le E-commerce. Tout comme
Twitter, les célébrités de la mode et du sport (ainsi que la téléréalité) sont très présentes sur Instagram,
entre autre, dans le but de faire suivre leurs activités à leurs publics. Les plateformes vous permettent
de créer vos contenus sous forme d’images, de vidéos, mais aussi d’aller en direct pour transmettre
directement l’image d’un évènement.
\end{itemize}

\section{Conclusion}
Les réseaux sociaux de discussion électronique ont un grand potentiel pour faciliter
l'apprentissage et favoriser la motivation des étudiants. Il est possible de l'utiliser pour les
soutenir dans leur démarche d'apprentissage et pour, plus spécifiquement, réaliser différentes
activités d'apprentissage qui impliquent de la communication et de la collaboration. C'est à vous
d'évaluer, selon votre contexte, si le réseau social est un outil approprié pour répondre à vos
besoins et à ceux de vos étudiants. \newline
Le réseau social est un outil accessible pour des enseignants et des tuteurs du collégial ainsi que
pour leurs étudiants. Il n'est pas difficile à utiliser, mais il ne fonctionne pas « tout seul ». Pour
que les étudiants y participent et que les objectifs pédagogiques visés puissent être atteints, vous
devez le planifier puis l'animer. C'est sûrement plus sage de l'expérimenter en commençant par
une activité simple… Avec l'expérience, vous verrez facilement comment vous y prendre pour
l'utiliser efficacement, dans votre contexte, pour de plus grands projets. Rappelez-vous aussi
qu'il est souvent avantageux d'utiliser, complémentairement au forum, d'autres outils pour
faciliter l'apprentissage, la collaboration et la motivation. \newline

\end{document}